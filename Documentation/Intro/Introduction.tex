\section{Introduction}
\subsection{Contexte}
La découverte de nouveaux matériaux joue un rôle central dans l'accélération de l'innovation technologique, en particulier dans le domaine de l'efficacité énergétique. 
Alors que la demande mondiale de solutions énergétiques durables augmente, les cellules solaires organiques (OSC) apparaissent comme une alternative prometteuse aux cellules photovoltaïques traditionnelles. 
Ces dispositifs se distinguent par leurs nombreux avantages, notamment leur faible coût de fabrication, leur flexibilité mécanique, leur facilité de production à grande échelle et leur faible toxicité. Cependant, l'amélioration de l'efficacité de la conversion de puissance (PCE) reste un défi majeur.

La qualité des matériaux constituant la couche active des OSC, composée de matériaux donneurs (polymères et petites molécules) et d'accepteurs, est un facteur décisif de leur performance. La découverte et l'optimisation de matériaux donneurs performants nécessitent des cycles expérimentaux longs, coûteux et gourmands en ressources, ce qui entrave l'adoption généralisée de cette technologie.

Dans ce contexte, l'intégration de techniques d'apprentissage automatique, en particulier l'apprentissage profond, offre une solution innovante. Celles-ci peuvent modéliser des données complexes et temporelles, ce qui les rend particulièrement adaptées à la prédiction des propriétés des matériaux et à la génération de nouvelles structures moléculaires optimisées pour maximiser le PCE.

\subsection{Objectifs du projet}
L'objectif principale de ce projet est d'accélérer la découverte de nouveaux matériaux donneurs pour les OSC en utilisant des techniques d'apprentissage automatique. 
En particulier, nous visons à développer un modèle basé sur l'apprentissage profond capable de prédire efficacement le PCE des matériaux donneurs. 
En parallèle, nous utiliserons des algorithmes de génération moléculaire pour créer de nouvelles structures moléculaires optimisées pour les OSC.
En vu du développement de ce projet, nous écrivons un rapport technique dans le but de permettre à quiconque de prendre en main le modèle d'apprentissage développé et de l'utiliser pour prédire les propriétés des matériaux donneurs, en particulier le PCE.